\documentclass[12pt,a4paper]{article}
\usepackage[utf8]{inputenc}
\usepackage{amsmath}
\usepackage{amsfonts}
\usepackage{amssymb}
\usepackage{graphicx}
\usepackage{tikz}
\usetikzlibrary{automata, positioning, arrows}

\title{ClasePractica1}
\author{Sherlyn Ballestero Cruz }

\begin{document}

\maketitle
\newpage
\textbf{1. Dq el conjunto de todos los n\'umeros primos es infinto.}
\\Supongamos que el conjunto $P=\{p_{1},p_{2},...,p_{n}\}$ que contiene  todos los n\'umeros primos es finito.\\
Se conoce que todo \'umero tiene al menso un divisor ptimo, por teor\'ia de n\'umeros.
\\Sea el n\'umero $p_{1}p_{2}...p_{n}+1$, el resto de dividir este n\'umero por cualquiera de los n primos comprendidos en P es 1,por tanto existe un primo no contenido en P que divide a este n\'umero. Contradicci\'on!!!!
\\\textbf{2. Dq un n\'umero es divisible por 3 si y solo si la suma de sus d\'igitos es divisible por 3.}
 Tengamos en cuenta que:\\
 $10\equiv 1(3)$\\
 $10^{i}\equiv 1(3)$\\
 $a^{i}\equiv 1(3)$\\
 $10^{i}a^{i}\equiv a^{i}(3)$\\
Dado que todo n\'umero se puede escribir como:
$d_{n}10^{n}...d_{1}=\sum_{i=0}^{n}d_{i}10^{i}$ y que
$\sum_{i=0}^{n}d_{i}10^{i} \equiv \sum_{i=0}^{n}d_{i} (3)$
por el analisis inicial, queda demostrado que todo n\'umero deja resto con tres lo mismo que deja la suma de sus d\'igitos.

\\\textbf{3. Sea $S$ un conjunto finito de enteros positivos. Asumiendo que hay exactamente 2023 pares ordenados $(x,y)$ en $S\times S$ tal que el producto $xy$ es un cuadrado perfecto. Pruebe que es posible encontrar al menos 4 elementos distintos en $S$ tales que ninguno de sus productos dos a dos sea un cuadrado perfecto.}\\
Sea $P$ el conjunto de los pares $<x,y>$ tal que $xy=a^{2}$ para alg\'un a y $x,y \in S$
\\Tengamos en cuenta que:\\
 $<x,x> \in P, \forall x \in S $pues $xx=x^{2} \Rightarrow P $ es reflexivo.
 Notese tambi\'en que $<x,y> \in P$ si $xy=a^{2}$ y $xy=yx=a^{2}$ $\Rightarrow <y,x> \in P $. Por tanto se cumple la simetr\'ia.\\
Adem\'as Si $<x,y \in P$ y $<x,z \in P \Rightarrow$\\
$xy=a^{2}$\\
$xz=b^{2}$\\
$xy^{2}z=a^{2}b^{2}$\\
$xy^{2}z=a^{2}b^{2}$\\
$xz=\frac{a^{2}b^{2}}{y^{2}}$\\
Luego $<x,y> \in P$ por tanto se cumple la transitividad.\\
Nos encontramos ante una relaci\'on de equivalencia.Luego si establecemos $A/R$ se puede observar que: en cada clase de equivalencia encontraremos los elementos que estaran rlacionados unos conotro en la relaci\'on y si tomo dos elementos de clases distintas cuando los multiplique no habr\'a manera que este de $a^{2}$ para alg\'un a, de pasar esto entonces el par perteneciese a la misma clase. Basta probar que se deben formar al menso cuatro clases de equivalencia.\\
Demostremos que en $S/R$ tiene al menos 4 clases de equivalencia.\\
Observemos que como $|P|=2023$ y que dado cada clase de equivalencia con x cantidad de elementos, como todos estan relacionados con todos esos x elementos representan $x^{2}$ de los 2023 pares, si solo hubiese una clase de equivalencia, pudiesemos observar una clara contradicci\'on dada por que 2023 no es un cuadrado.\\
Suponiendo que hay dos clases de equivalencias entonces se podr\'ia decir que: $i^{2}+j^{2}=2023$, siendo i y j las cardinalidades de ambas clases. Pero si analizamos la congruencia con 4 de los cuadrados , se obtiene que estos dejan resto 1 o 0 y que 2023 deja resto 3 los que nos muestra que es imposible obtener 2023 a opartir de sumar dos cuadrados.\\
Supongamos que existen tres clases de equivalencia, aqui veremos la contradicci\'on a partir de los restos que dejan con 8 dichos cuadrados, 2023 deja resto 7, y los cuadrados dejan resto, 1,4 o 0 , no existe manera con estos  tres n\'umerosa obtener la suma de 7 por lo que es imposible que hayan exactamente 3 clases de equivalencia.\\
Veamos que existe al menos una combinaci\'on de 4 cuadrados que me dar\'an como resulktado 2023, los que fueron posible encontrar despu\'es de una exaustiva busquedad: $42^[2]+15^{2}+5^{2}+3^{2}=2023 $.
\\
\textbf{4. Dq de un conjunto de premisas l\'ogicas donde existe una contradicci\'on se puede deducir cualquier proposici\'on.}
$T\vdas A  \curlywedge   \rightharpoondown A$\\
$T\vdash 0$\\
$T\vdash 0 \curlyvee q$\\
$T\vdash1\Rightarrow q$\\
$T\vdash q$\\  



\\\textbf{5. Dq, dado que $M$ contiene a todos los conjuntos que no se contienen a si mismos, $M$ no es un conjunto.}\\
 Sea $M$ el conjunto que contiene a todos los conjuntos que no se contienen a si mismos.\\
 Si M no se contiene a si mismo \Rightarrow $M \in M$ (Contradicci\'on)
 Si M se contiene \Rightarrow por la definici\'on de M se cumple que $M \in M$ (Contradicci\'on)
 Luego por reducci\'on al absurdo se puede concluir con que M no es un conjunto.\\
 

\\\textbf{6. Dq, dado un grafo $G$, si existe un camino de $u$ a $v$ entonces existe un camino simple de $u$ a $v$.}
Sea $C$ el conjunto que contiene las distancias de los caminos de u a v.\\
Existe camino de u a v por datos.\\
Luego C no est\'a vacio y las distancias representan son enteros no negativos,por ley del buen ordenamiento existe elemento minimo m.\\
Sea $u,....v$ un camino de longitud m de u a v en el grafo. Sup[ongamos que existe un ciclo, si eliminamos los elementos del ciclo aun podremos llegar de u a v y este nuevo camino tiene longitud inferior a m (contradicci\'on). \\ Se puede deducir que existe camino simple de u a v.
 

\\\textbf{7. Se quiere dar un vuelto de una cantidad de dinero $x, x > 8$. Dq es posible realizarlo usando s\'lo denominaciones de billetes de $3$ y $5$ pesos.}
Demostremos por inducci\'on.\\
Caso Base:
$x=9=3y+5z$\\
$y=3 y z=0$\\
Supongamos que se cumple para x=n\\
Para $z\neq 0$:\\
$n=3y+5Z$
$n+1=3y + 5z +1$\\
de donde:\\
$5z-2=3+5(z-1)$\\
luego: como 0=2-2\\
$n+1=3y + 5z +1 + 0$\\
$n+1=3y + 5z +1 +2-2$\\
$n+1=3y + (5z-2) +1 +2$\\
$n+1=3y + 3+5(z-1) +3$\\
$n+1=3(y+2)+5(z-1)$\\
$\Rightarrow$ Se cumple para todo n.
\\ Veamos para z=0\\
$n=3j, n>= 8,j>=3 $\\
$n+1=5*2+3(j-3)$\\
$\Rightarrow$ Se cumple para todo n.

\\\textbf{8. Dq, dado un grafo $G$, y un camino $P$ de $a$ a $b$ y un camino $Q$ de $a$ a $b$, con $P \neq Q$, entonces existe un ciclo en $G$.}
Demostremos por Inducci\'on.\\
Caso Base:Para tres nodos se cumple, verificable\\
Supongamos que se cumple $\forall m, 3=<m<n$\\
Supongamos que hay una arista com\'un para ambos caminos, si retiramos dicha arista entonces caemos en uno de los casos hip\'otesis.\\
Si la arista no es com\'un entonces, sean los caminos: $a,p_{1}...b$ y $a,p_{2}...b$, como existe camino de $p_{1} a a$ por teorema demosttrado anteriormente se tiene que existe camino simple, y con la arista $<a,p_{1}>$ se completa el ciclo.\\
\textbf{Ejercicios de razonamiento l\'ogico}\\
\textbf{3. Demuestre que el algoritmo de ordenaci\'on por m\'inimos sucesivos siempre produce un array ordenado.}\\
Supongamos que luego de aplicar el algoritmo a determinado array este no queda ordenado, donde dados dos elementos $a_{k} > a_{k+1}$ obtenemos que queda primero el elemento $a_{k}$, ubicandonos en el algorimo en el momento en que se analiza el elemneto correspondiente a la posici\'on donde se encuentra el elemento $a_{k}$, es hora de decidir el elemento correspondiente y este debe seleccionar el m\'inimo de todos los elementos que aun no se han insertado, pero como
$a_{k} > a_{k+1}$ y $a_{k+1}$ no se ha insertado entonces en dicho momento $a_{k}$ no ser\'a el elemento m\'inimo.\\

\textbf{4.Demuestre que el principio de inducción y el principio de buen ordenamiento son equivalentes.}
$PIM \Rightarrow PBO$\\
En un conjunto de enteros positivos diferente del vac\'io, con un elemento el \'inimo existe y es precisamente ese elemento.\\
Supongamos que en todo conjunto de enteros no negativos, de cardinalidad n tiene elemento m\'inimo.\\
Sea un conjunto arbitrario de cardinalidad n+1, tomamos un elemento al azar,. ahora se tiene un conjunto de cardinalidad n y este tiene elemento m\'inimo, lo comparamos con el que se tom\'o y nos quedamos con el menor, luego tambie\'en en n+1 existe el m\'inimo.\\
$PBO \Rightarrow PIM$\\
Sea $A \noq vacio$, el conjunto de todos los elementos que no cumplen determinada propiedad, y adem\'as se sabe que se cumple $P(1)$ y $P(n)\Rightarrow P(n+1)$, $\forall n>=1$
Existe el m\'inimo por POB, sea m,luego:\\
$m>=1$\\
$m>2$\\
$m-1>1$
$\Rightarrow P(m-1)$\\
Por tanto como $P(n)\Rightarrow P(n+1)$ se tiene $P(m)$, Contradicci\'on
 
\end{document}