\documentclass[12pt,a4paper]{article}
\usepackage[utf8]{inputenc}
\usepackage{amsmath}
\usepackage{amsfonts}
\usepackage{amssymb}
\usepackage{graphicx}
\usepackage{tikz}
\usetikzlibrary{automata, positioning, arrows}

\title{Notas de conferencias}
\author{Sherlyn Ballestero Cruz \\ Maria de Lourdes Choy Ferna\'ndez}

\begin{document}
\maketitle
\newpage
\pagestyle{myheadings}
\markright{Introdicci\'on a la Teor\'ia de lenguajes y Aut\'omatas}
\textit{Conferencia4}\\
 Hasta ahora hemos visto que son lenguajes regulares aquellos que son aceptados por alg\'un DFA,que los DFA son equivalente a los NFA, asi como a partir de las relaciones entre conjunto aplicado a lenguajes regulares se obtienen lenguajes regulares.\\
 Es decir,dado que tenemos un Lenguaje si encontramos un DFA o un NFA entonces sabemos que es regular, pero... ¿C\'omo decimos que un lenguaje no puede ser representado por alg\'un aut\'omata?...\\
\section{Lema del Bombeo}
\textbf{Lema del Bombeo:}
Sea L un lenguaje regular, existe un n (que depende de L), tal que,
$\forall$ w con  $\vert w \vert \geqslant n$, ω se puede escribir como
xyz, tal que:\\
1. y \neq \varepsilon\\
2. $\vert xy \vert\leqslant n$\\
3. $\forall k, xy^{k}z \in L$\\
\textbf{eh????}\\
Esto quiere decir que se puede encontrar la cadena  
$y \neq \varepsilon$, no muy lejana del inicio de w, que puede ser bombeada, o sea la podemos eliminar o repetir tantas veces como querramos y la cadena resultante $w'$ va a pertenecer al lenguaje.\\
\textbf{Demostraci\'on}\\
Supongamos que L es regular.\\
Se tiene que L=L(A) para alg\'un DFA A.\\
Supongamos que A tiene n estados.\\
Sea w,  $\vert w \vert \geqslant n$.\\
Luego $w=a_{1}a_{2}...a_{m}$, donde $m\geqslant n$ y $ \forall$ $a_{i}, a_{i}\in V$, o sea es un simbolo de entrada cada $a_{i}$.
Para i=1...n, se definen los estados $p_{i}$ de $\widehat{\delta}(q_{0},a_{1}a_{2}...a{i})$, donde \delta es la funci\'on de transicci\'on de A y $q_{0}$ el estado inicial.\\
Luego A est\'a en el estado $p_{i}$ despu\'es de leer los primerosa i s\'imbolos de w.\\
Notese que $q_{0}=p_{0}$.\\
Por principio del palomar no es posible tener n+1 $p_{i}$,pues  los $p_{i}$ para i=1...n, son distintos porque solo hay n diferentes estados.  Luego se pueden encontrar dos enteros i,j tal que $0\leqslant i\leqslant j \leqslant n$ tal que $p_{i}=p_{j}$.\\
1.$x=a_{1}...a_{i}$\\
2. $y=a_{i+1}...a_{j}$
3.$z=a_{j+1}...a_{m}$.\\
Lo que sucede es que x nos lleva hasta $p_{i}$, luego con y se pasa por una serie de estados y se llega nuevamente a $p_{i}$ y con z se termina w.\\
Si x es la cadena vacia, se parte del estado inicial y cuando se recorre y se vuelve a caer en el estado inicial, si z es vac\'io ser\'ia j=n=m.\\ 



\end{document}